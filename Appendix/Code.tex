\hiddensubsection{globals.py}
\begin{minted}{python}
#!/usr/bin/env python

def init():
	global ArmKeyPressed
	global ArmPin
	global SirenGPIOPin
	global StatusLED
	global modechange
        global timechange
        global maincount
        global param
        global mode
        global door1
        global window1, window2, light1

        maincount = 0
	modechange = 1
	ArmPin = 35
	SirenGPIOPin = 36
	StatusLED = 37

        mode = "Armed"
        door1 = 0
        window1 = 1
        window2 = 1
        light1 = 1
\end{minted}
        
\hiddensubsection{alarm.py}
\begin{minted}{python}
import RPi.GPIO as GPIO
import time
import serial
import globals
from flask import Flask, request, render_template
from multiprocessing import Process, Pipe, Queue, current_process
import subprocess

global parent_conn

app = Flask(__name__)

@app.route("/", methods=['Get', 'POST'])
def index():
	if request.method == 'POST': # Security System Sensor Input from a virtual website
		if request.form['mode_btn'] == 'ARM':
			globals.mode = "Armed"
			print "why is that not working"
		elif request.form['mode_btn'] == 'DisARM':
			globals.mode = "DisArmed"
			print "should DisArmed"
		elif request.form['mode_btn'] == 'Leaving':
			globals.mode = "Leaving"
		elif request.form['mode_btn'] == 'door1_open':
			globals.door1 = 0
		elif request.form['mode_btn'] == 'door1_closed':	
			globals.door1 = 1
		elif request.form['mode_btn'] == 'window1_open':
			globals.window1 = 0
		elif request.form['mode_btn'] == 'window1_closed':	
			globals.window1 = 1
		elif request.form['mode_btn'] == 'window2_open':
			globals.window2 = 0
		elif request.form['mode_btn'] == 'window2_closed':	
			globals.window2 = 1
		elif request.form['mode_btn'] == 'light1_on':
			globals.light1 = 0
		elif request.form['mode_btn'] == 'light1_off':
			globals.light1 = 1
		globals.modechange = 1
	return render_template('index.html')


def Checkinput():	
	if RFID.readable:
		serstr = RFID.readline()
		idstr = serstr[-6:-2]
		if idstr == 'B178':
			globals.mode = "DisArmed"
			globals.modechange = 1
	
	if blueT.readable and not globals.mode == "Guard" and not globals.mode == "ALARM":
		senstr = blueT.readline()
		if senstr == 'LEAVING#':
			globals.mode = "Leaving"
			globals.modechange = 1			

def update_mode():	
	print "Running update"										#update basic LED
	globals.timechange = time.time()
	globals.modechange = 0
	if globals.mode == "Armed":
		blueT.write('H3#')
		GPIO.output(37, True)
	elif globals.mode == "DisArmed":
		blueT.write('H0#')
		blueT.write('ALERTOFF#')
		GPIO.output(37, False)
	elif globals.mode == "ALARM":
		blueT.write('H1UNKNOWN#')	
		blueT.write('ALERT#')
		print "ALERT!!!!"

def IsAllLock():
	return globals.door1 and globals.window1 and globals.window2

def mainControl(conn):
	globals.init()
	global RFID
	global blueT
	GPIO.setmode(GPIO.BOARD)
	GPIO.setup(globals.SirenGPIOPin, GPIO.OUT)
	GPIO.output(globals.SirenGPIOPin,True)
	GPIO.setup(globals.ArmPin, GPIO.IN) 		#Arm pin setup
	GPIO.setup(globals.StatusLED, GPIO.OUT) 		#Arm pin setup
	GPIO.output(globals.StatusLED, True)
	RFID = serial.Serial('/dev/ttyUSB0', baudrate=9600, bytesize = 8, parity = 'N', stopbits = 1, timeout = 0.05)
	blueT = serial.Serial( '/dev/ttyAMA0', baudrate = 9600, parity='N', stopbits= 1, bytesize=8, timeout=0.05)

	# p = current_process()
	# print 'Starting:', p.name, p.pid
	while (True):

		globals.maincount+= 1
		globals.maincount = globals.maincount % 2
		if globals.maincount == 1:
			print "in mainloop %s" %str(globals.mode)		

		if globals.modechange:
			update_mode()

		if globals.mode =="Leaving":
			if not IsAllLock():	
				print "red"	
				Msg = "Door1 is not closed"
				blueT.write('H1%s#' %(Msg))		
			elif not globals.light1:
				print "yellow"
				Msg = "System is taking care of your home"
				blueT.write('H2%s#' %(Msg))
				if time.time() - globals.timechange > 5:
					globals.light1 = 1
			else:
				blueT.write('H3#')
				if time.time() - globals.timechange > 5:
					globals.mode = "Armed"
					globals.modechange = 1

		if globals.mode == "Armed":
			if not IsAllLock():
				globals.mode = "Guard"
				globals.modechange = 1
		if globals.mode == "Guard":
			lapse = time.time() - globals.timechange
			if lapse*10%10 < 5:
				blueT.write('H3#')
			elif lapse*10%10 > 5:
				blueT.write('H0#')
			if lapse > 10:
				globals.mode = "ALARM"
				globals.modechange = 1

		if globals.mode == "ALARM":
			lapse = time.time() - globals.timechange
			if lapse < 0.5:
				blueT.write('H1UNKNOWN#')
			elif lapse < 1.0:
				blueT.write('H0#')
			else:
				globals.timechange = time.time()

		Checkinput()
		time.sleep(0.2)		



if __name__ == '__main__':

	parent_conn, child_conn = Pipe()
	p = Process(name = "mainControl", target=mainControl, args=(child_conn, ))
	p.daemon = True
	p.start()
	# parent_conn.send(param)
	app.run(host='0.0.0.0', debug=False)
\end{minted}